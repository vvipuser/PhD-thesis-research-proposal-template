% !Mode:: "TeX:UTF-8"

\section{学位论文预期创新点}
%(要根据研究内容和国内外研究现状准确提炼,充分体现创新性)
%创新

\noindent \textbf{(1) ***}

1. 。

2. 。

\noindent \textbf{(2) ***}

。

\noindent \textbf{(3) ***}

1. 。


2. 。

\noindent \textbf{(4) **}

深度强化学习所依据的物理引擎并未考虑太多执行机构的参数性能,导致学习得到的步态在实际平台中行走效果变差或失效。通过引入机器人关节数据、控制参量等信息到物理引擎中,使深度学习得到的结果能贴近实际环境,得到能够实用的步态。


