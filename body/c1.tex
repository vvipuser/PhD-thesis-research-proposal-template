% !Mode:: "TeX:UTF-8"
%博士学位论文开题是开展学位论文工作的基础,是保证学位论文质量的重要环节。
%开题报告是留学博士生在导师指导下撰写并由导师审查批准的学术文件。准备开题过程是导师对博士生进行课题指导的重要步骤,也是师生在所选课题范围内共同切磋,整理、确定论文思路及主线的重要科学活动。
%开题报告是博士生向由本学科专家组成的评审小组汇报博士学位论文的选题依据、研究内容及研究方案等,即汇报博士学位论文“为什么做?做什么?怎么做?”。由本学科专家进行集体审议,检查学位论文选题是否正确、研究内容是否恰当、研究方案是否合理,同时也检查博士生对拟进行的研究题目理解是否深入、对相关研究领域研究现状了解是否全面、为进行课题研究所做的主观与客观上的准备是否充分等。在此基础上,评审专家还将从不同侧面、不同角度对论文的科学思路、研究方法等重要问题提供咨询、建议和帮助,使论文工作的方向、内容和方案更为合理。

\section{课题的来源及研究的目的和意义}
\subsection{课题来源或研究背景}

\subsection{研究的目的及意义}
%(不少于1000字)
