% !Mode:: "TeX:UTF-8"
\setlength{\subfigbottomskip}{0pt}
\CTEXoptions[bibname={主要参考文献}]
\CTEXsetup[name={,},number={}]{chapter}
\captionsetup{labelsep=space,font=small,justification=centering}
\arraycolsep=1.7pt
\graphicspath{{figures/}}
\renewcommand{\subcapsize}{\zihao{5}}
\renewcommand{\thesubfigure}{\alph{subfigure})}
\setcounter{secnumdepth}{4}
\newcommand{\pozhehao}{\raisebox{0.1em}{------}}
\titleformat{\chapter}{\center\zihao{-2}\heiti}{\chaptertitlename}{0.5em}{}
\titlespacing{\chapter}{0pt}{-4.5mm}{8mm}
\titleformat{\section}{\zihao{-3}\heiti}{\thesection}{0.5em}{}
\titlespacing{\section}{0pt}{4.5mm}{4.5mm}
\titleformat{\subsection}{\zihao{4}\heiti}{\thesubsection}{0.5em}{}
\titlespacing{\subsection}{0pt}{4mm}{4mm}
\titleformat{\subsubsection}{\zihao{-4}\heiti}{\thesubsubsection}{0.5em}{}
\titlespacing{\subsubsection}{0pt}{0pt}{0pt}
\makeatletter
\renewcommand\thesection{\@arabic \c@section} % 前面不带 thechapter
\makeatother

\theoremstyle{plain}
\theorembodyfont{\songti\rmfamily}
\theoremheaderfont{\heiti\rmfamily}
\newtheorem{definition}{\heiti 定义}
\newtheorem{example}{\heiti 例}
\newtheorem{algo}{\heiti 算法}
\newtheorem{theorem}{\heiti 定理}
\newtheorem{axiom}{\heiti 公理}
\newtheorem{proposition}{\heiti 命题}
\newtheorem{lemma}{\heiti 引理}
\newtheorem{corollary}{\heiti 推论}
\newtheorem{remark}{\heiti 注解}
\newenvironment{proof}{\noindent{\heiti 证明:}}{\hfill $ \square $ \vskip 4mm}
\theoremsymbol{$\square$}



% 定义页眉和页脚 使用fancyhdr 宏包
\newcommand{\makeheadrule}{
\rule[7pt]{\textwidth}{0.75pt} \\[-23pt]
\rule{\textwidth}{2.25pt}}
\renewcommand{\headrule}{
    {\if@fancyplain\let\headrulewidth\plainheadrulewidth\fi
     \makeheadrule}}
\makeatother
		
\pagestyle{fancyplain}
\renewcommand{\chaptermark}[1]{\relax}
\renewcommand{\sectionmark}[1]{\markright{#1}}
\fancyhf{}
\ifxueweidoctor
  \fancyhead[CO]{\songti \zihao{-5}\rightmark}
  \fancyhead[CE]{\songti \zihao{-5} 哈尔滨工业大学博士学位论文开题报告}%
  \fancyfoot[C]{\zihao{-5} -~\thepage~-}
	\renewcommand\bibsection{\section*{\centerline{\bibname}}
	\markboth{哈尔滨工业大学博士学位论文开题报告}{\bibname}}
\fi
\ifxueweimaster
  \fancyhead[C]{\songti \zihao{-5} 哈尔滨工业大学硕士学位论文开题报告}
  \fancyfoot[C]{\zihao{-5} -~\thepage~-}
	\renewcommand\bibsection{\section*{\centerline{\bibname}}
	\markboth{哈尔滨工业大学硕士学位论文开题报告}{\bibname}}
\fi

\renewcommand{\CJKglue}{\hskip 0.56pt plus 0.08\baselineskip} %加大字间距,使每行33个字
\def\defaultfont{\renewcommand{\baselinestretch}{1.62}\normalsize\selectfont}
% 调整罗列环境的布局
\setitemize{leftmargin=3em,itemsep=0em,partopsep=0em,parsep=0em,topsep=-0em}
\setenumerate{leftmargin=3em,itemsep=0em,partopsep=0em,parsep=0em,topsep=0em}
\renewcommand{\theequation}{\arabic{equation}}
\renewcommand{\thetable}{\arabic{table}}
\renewcommand{\thefigure}{\arabic{figure}}

\makeatletter
\renewcommand{\p@subfigure}{\thefigure~}
\makeatother

\newcommand{\citeup}[1]{\textsuperscript{\cite{#1}}} % for WinEdt users

% 封面、摘要、版权、致谢格式定义
\makeatletter
\def\title#1{\def\@title{#1}}\def\@title{}
\def\titlesec#1{\def\@titlesec{& \rule[-4pt]{200pt}{1pt}\hspace{-326pt}\centerline{\textbf{#1}}}}\def\@titlesec{}
\def\affil#1{\def\@affil{#1}}\def\@affil{}
\def\subject#1{\def\@subject{#1}}\def\@subject{}
\def\author#1{\def\@author{#1}}\def\@author{}
\def\bdate#1{\def\@bdate{#1}}\def\@bdate{}
\def\supervisor#1{\def\@supervisor{#1}}\def\@supervisor{}
\def\assosupervisor#1{\def\@assosupervisor{\textbf{副\hfill 导\hfill 师} & \rule[-4pt]{200pt}{1pt}\hspace{-326pt}\centerline{\textbf {#1}}\\}}\def\@assosupervisor{}
\def\cosupervisor#1{\def\@cosupervisor{\textbf{联\hfill 合\hfill 导\hfill 师} & \rule[-4pt]{200pt}{1pt}\hspace{-326pt}\centerline{\textbf {#1}}\\}}\def\@cosupervisor{}
\def\date#1{\def\@date{#1}}\def\@date{}
\def\stuno#1{\def\@stuno{#1}}\def\@stuno{}
% 定义封面
\ifxueweidoctor
\def\makecover{
    \thispagestyle{empty}
    \zihao{-2}\vspace*{10mm}
		\renewcommand{\CJKglue}{\hskip 2pt plus 0.08\baselineskip}
    \centerline{\kaishu\textbf{哈尔滨工业大学}}
		\vspace{10mm}
		\centerline{\zihao{2}\songti\textbf{博士学位论文开题报告}}
		\vspace{10mm}
		\centerline{\zihao{3}\songti\textbf\@title}

		\renewcommand{\CJKglue}{\hskip 0pt plus 0.08\baselineskip}



    \zihao{3}\vspace{2\baselineskip}
    \hspace*{36pt}{\songti
	\renewcommand{\arraystretch}{1.3}
    \begin{tabular}{l@{}l}
    \textbf{院\hfill (系)}   & \rule[-4pt]{200pt}{1pt}\hspace{-326pt}\centerline{\textbf\@affil}\\
    \textbf{学\hfill 科}     & \rule[-4pt]{200pt}{1pt}\hspace{-326pt}\centerline{\textbf\@subject}\\
    \textbf{导\hfill 师}     & \rule[-4pt]{200pt}{1pt}\hspace{-326pt}\centerline{\textbf\@supervisor}\\
    \@assosupervisor
	\@cosupervisor
    \textbf{研\hfill 究\hfill 生}      & \rule[-4pt]{200pt}{1pt}\hspace{-326pt}\centerline{\textbf\@author}\\
    \textbf{学\hfill \hfill \hfill 号}  & \rule[-4pt]{200pt}{1pt}\hspace{-326pt}\centerline{\textbf\@stuno}\\
    \textbf{开题报告日期} & \rule[-4pt]{200pt}{1pt}\hspace{-326pt}\centerline{\textbf\@date}\\
%    \textbf{论\hfill 文\hfill 题\hfill 目}  & \rule[-4pt]{200pt}{1pt}\hspace{-326pt}\centerline{\textbf\@title}\\
%    \@titlesec
    \end{tabular}\renewcommand{\arraystretch}{1}}
	\vfill
    \centerline{\songti\textbf{研究生院制}}
    \centerline{\songti\textbf{二〇一五年三月}}

%%定义内封
%\newpage
%\thispagestyle{empty}
%\zihao{5}\vspace*{2em}
%\begin{center}
%  \heiti\zihao{3}说\hspace{3em}明
%\end{center}
%\vspace*{40pt}
%	\renewcommand{\arraystretch}{1.25}
%    {
%    \songti\zihao{5}
%    \hangindent=2em\noindent 一、开题报告应包括下列主要内容:
%    \begin{enumerate}[leftmargin=36pt]
%    \item 课题来源及研究的目的和意义;
%    \item 国内外在该方向的研究现状及分析(文献综述);
%    \item 前期的理论研究与试验论证工作的结果;
%    \item 学位论文的主要研究内容、实施方案及其可行性论证;
%    \item 论文进度安排,预期达到的目标;
%    \item 为完成课题已具备和所需的条件、外协计划及经费;
%    \item 预计研究过程中可能遇到的困难、问题,以及解决的途径;
%    \item 主要参考文献(应在~50~篇以上,其中外文资料不少于二分之一,参考文献中近五年内发表的文献一般不少于三分之一,且必须有近二年内发表的文献资料)。
%    \end{enumerate}
%    \noindent 二、开题报告字数应不少于~1.5~万字。
%
%    \noindent 三、开题报告时间应最迟应于第四学期结束前完成。
%
%    \hangindent=2em\noindent 四、若本次开题报告未通过,需在三个月内再次进行开题报告。第二次学位论文开题报告
%    仍未通过者,将取消其学籍。
%
%    \hangindent=2em\noindent 五、开题报告结束后,评议小组要填写《博士学位论文开题报告评议结果》上报院(系)研
%    究生教学秘书备案。
%
%    \noindent 六、此表不够填写时,可另加附页。
%    }
%	\renewcommand{\arraystretch}{1}
%    \clearpage
}
\fi

\ifxueweimaster
\def\makecover{
    \thispagestyle{empty}
    \zihao{-2}\vspace*{10mm}
		\renewcommand{\CJKglue}{\hskip 2pt plus 0.08\baselineskip}
    \centerline{\kaishu\textbf{哈尔滨工业大学}}
		
		\vspace{10mm}
		\centerline{\zihao{2}\songti\textbf{硕士学位论文开题报告}}

		\renewcommand{\CJKglue}{\hskip 0pt plus 0.08\baselineskip}
\vspace{30pt}
\zihao{-2}
\begin{center}\songti\textbf{题~目:\@title}\end{center}
\vspace{30pt}
    \zihao{3}
    \hspace*{68pt}{\songti
	\renewcommand{\arraystretch}{1.3}
    \begin{tabular}{l@{}l}
    \textbf{院\hfill (系)}   & \rule[-4pt]{200pt}{1pt}\hspace{-326pt}\centerline{\textbf\@affil}\\
    \textbf{学\hfill 科}     & \rule[-4pt]{200pt}{1pt}\hspace{-326pt}\centerline{\textbf\@subject}\\
    \textbf{导\hfill 师}     & \rule[-4pt]{200pt}{1pt}\hspace{-326pt}\centerline{\textbf\@supervisor}\\
    \@assosupervisor
	\@cosupervisor
    \textbf{研\hfill 究\hfill 生}      & \rule[-4pt]{200pt}{1pt}\hspace{-326pt}\centerline{\textbf\@author}\\
    \textbf{学\hfill 号}  & \rule[-4pt]{200pt}{1pt}\hspace{-326pt}\centerline{\textbf\@stuno}\\
    \textbf{开题报告日期} & \rule[-4pt]{200pt}{1pt}\hspace{-326pt}\centerline{\textbf\@date}\\
    \end{tabular}
		\renewcommand{\arraystretch}{1}}
	\vfill
    \centerline{\songti\textbf{研究生院培养处制}}

%%定义内封
%\newpage
%\thispagestyle{empty}
%\zihao{5}\vspace*{2em}
%\begin{center}
%  \heiti\zihao{3}说\hspace{3em}明
%\end{center}
%\vspace*{40pt}
%	\renewcommand{\arraystretch}{1.25}
%    {\songti\zihao{5}
%    \hangindent=2em
%	\noindent 一、开题报告应包括下列主要内容:
%    \begin{enumerate}[leftmargin=36pt]
%	\item 课题来源及研究的目的和意义;
%	\item 国内外在该方向的研究现状及分析;
%	\item 主要研究内容;
%	\item 研究方案及进度安排,预期达到的目标;
%	\item 为完成课题已具备和所需的条件和经费;
%	\item 预计研究过程中可能遇到的困难和问题,以及解决的措施;
%	\item 主要参考文献。
%    \end{enumerate}
%    \noindent 二、对开题报告的要求
%	\begin{enumerate}[leftmargin=36pt]
%	\item 开题报告的字数应在~5000~字以上;
%	\item 阅读的主要参考文献应在~20~篇以上,其中外文资料应不少于三分之一。硕士研究生应在导师的指导下着重查阅近年内发表的中、\hspace{-1pt} 外文期刊文章。\hspace{-1pt}本学科的基础和专业课教材一般不应列为参考资料。
%    \end{enumerate}
%    \noindent 三、开题报告时间应最迟不得超过第三学期的第三周末。
%
%    \hangindent=2em\noindent 四、如硕士生首次开题报告未通过,\hspace{-2pt}需在一个月内再进行一次。\hspace{-3pt}若仍不通过,\hspace{-2pt}则停止硕士论文工作。
%
%    \noindent 五、此表不够填写时,可另加附页。
%
%\hangindent=2em\noindent 六、开题报告进行后,此表同硕士学位论文开题报告评议结果存各系(院)研究生秘书书处,以备研究生院及所属学院进行检查。
%
%    }
%	\renewcommand{\arraystretch}{1}
%    \clearpage
}


\fi
\makeatother
