% !Mode:: "TeX:UTF-8"

\def\usewhat{pdflatex} % 定义编译方式 pdflatex, dvipdfmx, or xelatex


\def\xuewei{Doctor} % 定义学位 Doctor or Master
%\def\xueweidoctor
%bmeps .jpg .eps

\documentclass[cs4size,openany,UTF8]{ctexbook}
\usepackage[a4paper,text={160true mm,234true mm},top=30.5true mm,left=25true mm,head=5true mm,headsep=2.5true mm,foot=8.5true mm]{geometry}
\usepackage{booktabs}
\usepackage{longtable}
\usepackage{caption}
\usepackage{amsmath}
\usepackage{txfonts}
\usepackage{bm}
\usepackage{graphicx}
\usepackage{enumitem}
\usepackage{fancyhdr}
\usepackage{ntheorem}
\usepackage{titlesec}
\usepackage{subfigure}
\usepackage{tabularx}
\usepackage[sort&compress,numbers]{natbib}
\usepackage[boxed,linesnumbered,algochapter]{algorithm2e}
\usepackage{epstopdf}	
\usepackage{hyperref}
%\usepackage[unicode,dvipdfmx]{hyperref}

%`\usepackage{array,graphicx,subfig}

%\usepackage{array}
%\usepackage[caption=false,font=footnotesize,labelfont=sf,textfont=sf]{subfig}


\usepackage{enumerate}


\begin{document}

\newif\ifxueweidoctor
\newif\ifxueweimaster
\def\temp{Doctor}
\ifx\temp\xuewei
  \xueweidoctortrue  \xueweimasterfalse
\fi
\def\temp{Master}
\ifx\temp\xuewei
  \xueweidoctorfalse  \xueweimastertrue
\fi

\input{format}
% !Mode:: "TeX:UTF-8"

\affil{计算机科学与技术}
\subject{计算机科学与技术}
\author{作者}
\supervisor{导师}
% \assosupervisor{副导名}%若无副导师,请屏蔽掉此句
% \cosupervisor{联导名}%若无联合导师,请屏蔽掉此句
\date{20**年10月**日}
\stuno{**B**}
\bdate{20xx年xx月}

\ifxueweidoctor
  \title{题目~:~***} %论文题目,此处最多12个字,题目剩余的字放在\ctitlesec里面
%  \titlesec{关键技术的研究}
\fi
\ifxueweimaster
  \title{xxxx}
\fi

\makecover
\clearpage
\setcounter{page}{1}

\zihao{-4}


\tableofcontents
\thispagestyle{empty}
\clearpage{\pagestyle{empty}}
\setcounter{page}{1}

%开题报告字数应不少于1.5万字。
%

% !Mode:: "TeX:UTF-8"
%博士学位论文开题是开展学位论文工作的基础,是保证学位论文质量的重要环节。
%开题报告是留学博士生在导师指导下撰写并由导师审查批准的学术文件。准备开题过程是导师对博士生进行课题指导的重要步骤,也是师生在所选课题范围内共同切磋,整理、确定论文思路及主线的重要科学活动。
%开题报告是博士生向由本学科专家组成的评审小组汇报博士学位论文的选题依据、研究内容及研究方案等,即汇报博士学位论文“为什么做?做什么?怎么做?”。由本学科专家进行集体审议,检查学位论文选题是否正确、研究内容是否恰当、研究方案是否合理,同时也检查博士生对拟进行的研究题目理解是否深入、对相关研究领域研究现状了解是否全面、为进行课题研究所做的主观与客观上的准备是否充分等。在此基础上,评审专家还将从不同侧面、不同角度对论文的科学思路、研究方法等重要问题提供咨询、建议和帮助,使论文工作的方向、内容和方案更为合理。

\section{课题的来源及研究的目的和意义}
\subsection{课题来源或研究背景}

\subsection{研究的目的及意义}
%(不少于1000字)
%课题来源及研究的目的和意义
\newpage
% !Mode:: "TeX:UTF-8"

\section{国内外在该方向的研究现状及分析(文献综述)}
%(注意对所引用国内外文献的准确标注)
\subsection{国外研究现状}
\subsubsection{国外机器人研究现状}
\subsection{国内研究现状}
\subsubsection{国内机器人研究现状}

\subsection{国内外文献综述的简析}
%(不少于1000字)
%(综合评述:国内外研究取得的成果,存在的不足或有待深入研究的问题)
%国内外在该方向的研究现状及分析(文献综述)
\newpage
% !Mode:: "TeX:UTF-8"

\section{前期的理论研究与试验论证工作的结果}
%前期的理论研究与试验论证工作的结果
\newpage
% !Mode:: "TeX:UTF-8"

\section{学位论文的主要研究内容、实施方案及其可行性论证}
\subsection{主要研究内容}
%(不少于2000字)


\subsection{实施方案及其可行性论证}
%(不少于3000字)
%学位论文的主要研究内容、实施方案及其可行性论证
\newpage
% !Mode:: "TeX:UTF-8"

\section{论文进度安排,预期达到的目标}
\subsection{进度安排}
%(从确定博士选题收集文献写起)
2016.10$\sim$2016.12:。

2017.01$\sim$2017.10:。

2017.10$\sim$2018.12:。

2019.01$\sim$2019.06:。

2019.07$\sim$2019.12:。

2020.01$\sim$2020.07:撰写论文,准备答辩。


\subsection{预期达到的目标}

预期达到以下目标:

(1)

(2)

(3)

(4)

(5)

(6)

研究方案。
%论文进度安排,预期达到的目标
\newpage
% !Mode:: "TeX:UTF-8"

\section{学位论文预期创新点}
%(要根据研究内容和国内外研究现状准确提炼,充分体现创新性)
%创新

\noindent \textbf{(1) ***}

1. 。

2. 。

\noindent \textbf{(2) ***}

。

\noindent \textbf{(3) ***}

1. 。


2. 。

\noindent \textbf{(4) **}

深度强化学习所依据的物理引擎并未考虑太多执行机构的参数性能,导致学习得到的步态在实际平台中行走效果变差或失效。通过引入机器人关节数据、控制参量等信息到物理引擎中,使深度学习得到的结果能贴近实际环境,得到能够实用的步态。


%学位论文预期创新点(要根据研究内容和国内外研究现状准确提炼,充分体现创新性)
\newpage
% !Mode:: "TeX:UTF-8"

\section{为完成课题已具备和所需的条件、外协计划及经费}

目前已具备的实验条件包括

(1)	

(2)	

(3) 

(3)	

(4) 
%为完成课题已具备和所需的条件、外协计划及经费
\newpage
% !Mode:: "TeX:UTF-8"

\section{预计研究过程中可能遇到的困难、问题,以及解决的途径}

\subsection{预计研究过程中可能遇到的困难、问题}

本课题可能出现的困难和问题主要包括:



以上是可以预见的在研究过程中可能遇到的问题,实际上在研究过程中可能遇到一些其他的问题,但是从整体的大方向上判断,可预见的困难是可以通过努力克服的。

\subsection{解决的途径}

针对以上可以预见的不可避免的困难和技术难点,可以采取以下可行的策略解决问题,并继续课题的研究:

(1)。

(2)。

(3)。

%预计研究过程中可能遇到的困难、问题,以及解决的途径
\newpage
% !Mode:: "TeX:UTF-8"

\section{主要参考文献}
%(应在50篇以上,其中外文资料不少于二分之一,参考文献中近五年(从开题当年算起)内发表的文献一般不少于三分之一,且必须有近二年内发表的文献资料)。
%阅读的主要参考文献应在50篇以上,其中外文资料不少于二分之一,参考文献中近五年内发表的文献一般不少于三分之一,且必须有近二年内发表的文献资料。教材、技术标准、产品样本等一般不应列为参考文献。
\begin{thebibliography}{99}

\bibitem{1} Kajita  S,  Hirukawa  H,  Harada  K,  et  al.  Introduction  to  Humanoid  Robotics[M].  Berlin:  Springer Berlin Heidelberg, 2014.

\end{thebibliography}
%主要参考文献
\newpage


%
%\begin{thebibliography}{99}
%\bibitem{c1}    xx.xx[D]. xx:xx,2018.1.1.
%\end{thebibliography}


%\bibliographystyle{GBT7714-2005NLang-HIT}
%\addtolength{\bibsep}{-0.8em}
%\nocite{*}
%\bibliography{reference}

\end{document}
